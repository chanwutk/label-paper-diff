\section{Fast Overlap Detection with Occupancy Bitmaps}

As we discussed earlier, the bottleneck of a label placement algorithm is to detect whether existing elements overlap with the new label to be placed.
This section presents the \emph{occupancy bitmap}, a data structure that we use to accelerate overlap detection.

Our occupancy bitmap is responsible for three parts of a label placement algorithm.
First, it records occupied areas of existing marks that labels need to avoid.
Second, it checks if a position of a label is available for placing (overlap detection).
And third, it records occupied areas of placed labels (occupancy updates).

An occupancy bitmap is a two-dimensional bitmap of the same size as the chart.
Building on well-known bitmaps~\cite{wiki:bitarray}, each bit in the occupancy bitmap records the occupancy of its corresponding pixel in the chart as shown in
\autoref{fig:place_label}.
A bit is set to one if its corresponding pixel is occupied and zero otherwise.

Occupancy bitmaps provide two key benefits over the data structure used in Particle-Based labeling.
First, by using a bitmap to store occupancy information, the time required to check if placing a label at a certain position overlaps with any existing elements depends only on the chart size and label size, but does not depend on the complexity and the number of existing elements.
Second, the bitmap structure leverages bitwise operators to accelerate overlap detection and occupancy updates.
We implement the bitmap using a one-dimensional array of n-bits integers, in which each integer is a bitstring representing
a subpart of a row in the bit matrix.
Thus, an integer in the array encodes the occupancy of n-horizontally consecutive pixels in the chart.\footnote{In our JavaScript implementation, we use 32-bit integer as it is the largest available integer size in JavaScript.}
For a chart with width $w$ and height $h$, the occupancy of the pixel $(x, y)$
is the bit at the position $((y \times w) + x) \bmod n$ of the integer at the array index $\lfloor\frac{(y \times w) + x}{n}\rfloor$.
This bitmap layout is efficient because it supports looking up and updating a vector of bits simultaneously, instead of one bit at a time.
Both operations are performed on a rectangular area.
(1) The lookup operation checks if the area is partly occupied to decide whether the area is available for placing labels.
(2) The update operation marks all bits in the area taken up by a placed label as occupied.

In the underlying array of the bitmap, there are two sets of integer entries that interact with the areas.
First, $I_f$ is the set of integer entries that are fully covered by the area, shown in the red column number 1 and 2 in \autoref{fig:bitmask}.
Second, $I_p$ is the set of integer entries that are partly covered by the area, in the red column 0 and 4 in \autoref{fig:bitmask}.

For lookup, the algorithm can check if each integer entry in $I_f$ is zero.
For each entry in $I_p$, the algorithm masks the entry with a bitwise-and operation to include only the bits inside the area, before checking if the masking result is zero.
For example, $arr[5]$ and $arr[6]$ in \ \autoref{fig:bitmask} are in $I_f$. The integer value of each entry is $0000_2$, meaning that the four pixels it represents are all unoccupied.
$arr[4]$ and $arr[7]$ are in $I_p$.
The integer value of $arr[4]$ is $0000_2$.
The masking value is $0011_2$ because only the right two bits are in the area.
The masking result is $0000_2 \& 0011_2 = 0000_2$, meaning that the two bits inside the area, are all unoccupied.
The integer value of $arr[7]$ is $0011_2$.
The masking value is $1000_2$ because only the leftmost bits is in the area.
The masking result is $0011_2 \& 1000_2 = 0000_2$, meaning that the leftmost bit, which is inside the area, is unoccupied.
Then, we can conclude that the bits from coordinate $(2, 1)$ to $(12, 1)$ are all unoccupied (zero).
The process is repeated for row 1 to row 4 to check the whole rectangular area for the potential label position.

All the bits represented by each integer entry in $I_f$ can be set as occupied simultaneously by setting the integer value of each entry to $11...11_2$.
For each entry in $I_p$, the algorithm masks the entry with a bitwise-or operation to retain previous values of the bits outside of the area.
For the example shown in \autoref{fig:bitmask}, $arr[5]$ and $arr[6]$ are in $I_f$, each entry is set to $1111_2$, meaning that four bits that it represents are all set to occupied.
$arr[4]$ and $arr[7]$ are in $I_p$.
The integer value of $arr[4]$ is $0000_2$.
The masking value is $0011_2$ because only the right two bits are in the area.
The masking result is $0000_2 | 0011_2 = 0011_2$.
The entry $arr[4]$ is then set to $0011_2$, meaning that the two bits inside the area, are all set to occupied.
The integer value of $arr[7]$ is $0011_2$.
The masking value is $1000_2$ because only the leftmost bits are in the area.
The masking result is $0011_2 | 1000_2 = 1011_2$.
The entry $arr[7]$ is then set to $1011_2$, meaning that the leftmost bit, which is inside the area, is set to occupied.
Notice that the right three bits of $arr[7]$ are kept as they were because the algorithm masks the integer entry with $1000_2$ to retain their previous values.
After running the steps above, all bits from coordinate $(2, 1)$ to $(12, 1)$ are set to occupied.
However, the algorithm does not repeat the process for all rows 1 to 4.
Instead, it repeats for the first, the last, and every $labelHeight_{min}$ row where $labelHeight_{min}$ is the height of the label that has the shortest height.
So, if $labelHeight_{min}=2$, this process repeats for row 1, 3, and 4.
Updating fewer rows of bits speeds up update operations, while not losing any information about the area marked as occupied.
A label of at least height $labelHeight_{min}$ that overlaps with the occupied area is guaranteed to overlap with at least one of the rows set to occupied.

Checking for overlap or marking an integer entry to occupied can be done in a constant number of bitwise-operations.
These operations have constant runtime, regardless of the size of the integer.
Our implementation uses the largest available integer size, to process many bits in parallel.

To record the projection of the marks that labels need to avoid (marks in $M$).
First, we draw all the marks in $M$ into a canvas.
Every pixel that is not fully opaque is considered occupied and its corresponding bit in the bitmap set to one.
The number of bits used to represent graphical marks is bounded by the size of the chart.
Therefore, our labeling algorithm using the \emph{occupancy bitmap} can perform occupancy checks and updates consistently,
regardless of the number and size of the graphical marks that need to not overlap with labels.
