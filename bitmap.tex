\section{Fast Overlap Detection with Occupancy Bitmaps}

We now present an \emph{occupancy bitmap} as a data structure to accelerate overlap detection, which is the bottleneck of label placement.

To accelerate overlap detection, an occupancy bitmap allows a label placement algorithm to efficiently check if a candidate position for a new label is previously occupied.  
Once a new label is placed, the labeling algorithm can also quickly
update the occupancy bitmap to include the new occupied area. 

An occupancy bitmap is a two-dimensional bitmap of the same resolution as the screen-space (in pixel area) of the chart.
Building on well-known bitmap (or bit arrays) structures~\cite{wiki:bitarray}, each bit in the occupancy bitmap records the occupancy of its corresponding pixel in the chart as shown in
\autoref{fig:place_label}.
A bit is set to one if its corresponding pixel is occupied and zero otherwise.

Occupancy bitmaps provide two key benefits over the data structure used in Particle-Based Labeling.
First, by using a bitmap to store occupancy information, the time required to check if placing a label at a certain position overlaps with any existing elements depends only on the chart size and label size, but does not depend on the complexity and the number of existing elements.
Second, the bitmap structure leverages bitwise operators to accelerate two key operations for overlap detection: 
(1) The \emph{lookup} operation checks if the area is partly occupied to decide whether the area is available for placing labels;
(2) The \emph{update} operation marks all bits in the area taken up by a new label placed as occupied.

We implement the bitmap using a one-dimensional array of n-bits integers, in which each integer represents the bits of a contiguous subset of a row in the bit matrix.
Thus, an integer in the array encodes the occupancy of $n$ horizontally consecutive pixels in the chart.\footnote{In our JavaScript implementation, we use 32-bit integer as it is the largest available integer size in JavaScript.}
For a chart with width $w$ and height $h$, the occupancy of the pixel $(x, y)$
is the bit at the position $((y \times w) + x) \bmod n$ of the integer at the array index $\lfloor\frac{(y \times w) + x}{n}\rfloor$.
This bitmap layout is efficient because it supports looking up and updating a vector of bits simultaneously, instead of one bit at a time.

In the underlying array of the bitmap, there are two sets of integer entries that interact with the areas.
First, $I_f$ is the set of integer entries that are fully covered by the area, shown in the red column number 1 and 2 in \autoref{fig:bitmask}.
Second, $I_p$ is the set of integer entries that are partly covered by the area, in the red column 0 and 4 in \autoref{fig:bitmask}.

For lookup, the algorithm can check if each integer entry in $I_f$ is zero.
For each entry in $I_p$, the algorithm masks the entry with a bitwise-and operation to include only the bits inside the area, before checking if the masking result is zero.
For example, $arr_5$ and $arr_6$ in \ \autoref{fig:bitmask} are in $I_f$. The integer value of each entry is $0000_2$, meaning that the four pixels it represents are all unoccupied.
$arr_4$ and $arr_7$ are in $I_p$.
The integer value of $arr_7$ is $0011_2$.
The masking value is $1000_2$ because only the leftmost bit is in the area.
The masking result is $0011_2 \& 1000_2 = 0000_2$, meaning that the leftmost bit, which is inside the area, is unoccupied.
The same process with different masking value is applied for the integer value of $arr_4$.
Then, we can conclude that the bits from coordinate $(2, 1)$ to $(12, 1)$ are all unoccupied (zero).
The process is repeated for row 1 to row 4 to check the whole rectangular area for the potential label position.

All the bits represented by each integer entry in $I_f$ can be set as occupied simultaneously by setting the integer value of each entry to $11...11_2$.
For each entry in $I_p$, the algorithm masks the entry with a bitwise-or operation to retain previous values of the bits outside of the area.
For the example shown in \autoref{fig:bitmask}, $arr_5$ and $arr_6$ are in $I_f$, each entry is set to $1111_2$, meaning that four bits that it represents are all set to occupied.
$arr_4$ and $arr_7$ are in $I_p$.
The integer value of $arr_7$ is $0011_2$.
The masking value is $1000_2$ because only the leftmost bits are in the area.
The masking result is $0011_2 | 1000_2 = 1011_2$.
The entry $arr_7$ is then set to $1011_2$, meaning that the leftmost bit, which is inside the area, is set to occupied.
Notice that the right three bits of $arr_7$ are kept as they were because the algorithm masks the integer entry with $1000_2$ to retain their previous values.
The same process with different masking value is applied for the integer value of $arr_4$.
After running these steps, all bits from coordinate $(2, 1)$ to $(12, 1)$ are set to occupied.
However, the algorithm does not repeat the process for all rows 1 to 4.
Instead, it only marks the first, the last, and every $labelHeight_{min}$ row as occupied; $labelHeight_{min}$ is the height of the label that has the shortest height.
So, if $labelHeight_{min}=2$, this process repeats for row 1, 3, and 4.
Updating fewer rows of bits speeds up update operations, while not losing any information about the area marked as occupied.
A label of at least height $labelHeight_{min}$ that overlaps with the occupied area is guaranteed to overlap with at least one of the rows set to occupied.

Checking for overlap or marking an integer entry as occupied can be done in a constant number of bitwise-operations.
These operations have constant runtime, regardless of the size of the integer.
Our implementation uses the largest available integer size, to process many bits in parallel.

To record the areas of the marks for the labels to avoid, we rasterize all the marks in $M$ onto
the bitmap.
Every pixel that is not fully transparent is considered occupied and its corresponding bit in the bitmap set to one.
The number of bits used to represent marks is bounded by the size of the chart.
Thus, the runtime for rasterization linearly depends on the chart resolution and number of the graphical marks.
After the rasterization, a labeling algorithm using the \emph{occupancy bitmap} can efficiently perform occupancy checks and updates.
The runtime for an occupancy check or an update only depends linearly on the size of the label,
regardless of the number and size of the marks that need to not overlap with labels.