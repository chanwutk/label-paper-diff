\section{Fast Overlap Detection for Labeling Charts}

In this section, we apply fast overlap detection using the occupancy bitmap
to place labels in scatter and connected scatter plots, line charts, and maps.
The algorithm for placing labels is greedy, following the labeling steps described in \autoref{sec:related}.
It first rasterizes all marks onto an \emph{occupancy bitmap}.
It then places all labels in one pass. 
For each data point to label, the algorithm iterates through the candidate label positions.
It places the label at the first candidate position that does not overlap with any mark in the occupancy bitmap (skipping the remaining candidates).
Before continuing with the next label, it marks the area taken by the label placed as occupied in the occupancy bitmap 
by marking  the rectangular bounding box of the label (\autoref{fig:bitmask}).
The algorithm to add labels in the these example chart types only differs in terms of (1) the graphical marks to be avoided by labels and (2) the candidate positions for labels.

For scatter and connected scatter plots, we use the 8-position model~\cite{imhof1975positioning} to generate candidate positions around each point.
For scatter plots, the marks to be avoided by labels include the point marks that represent records in the plot.
For connected scatter plots, the marks include the points that represent records in the plots and the lines that connect them (\autoref{fig:connected_scatter}).

In a line chart, each group of line includes a series of points and a path that connects all the points.
Line charts are similar to connected scatter plots, but often one label represents a whole line instead of a single record.
Therefore, the labeling algorithm may place one label per line, at the end of the line it represents.
In this case, candidate positions include top-right, right, and bottom-right of the rightmost point of each line.

As shown in \autoref{fig:eval_airport_vis}, a map can contain points that represent locations, which need to be labelled, and paths that represent geographical features (\eg country outlines).
In this example, we also draw line segments to show paths between different locations.
Similar to scatter plots, we use the 8-position model to generate candidate positions for maps. 