\section{Applications to Labeling Algorithms}

In this section, we apply fast overlap detection using the occupancy bitmap
to place labels in scatter plots, line charts, and cartographic maps.
The algorithm for placing labels is greedy, following the labeling steps mentioned in the related work section.
It first projects all the graphical marks to be avoided by the labels into an \emph{occupancy bitmap}.
It then looks through each of the data points in one pass.
For each data point, the algorithm iterates through the candidate positions of the label to be placed.
It places the label at the first candidate position that does not overlap with any graphical mark in the occupancy bitmap (skipping the rest of the candidate locations).
To place the label, it marks the rectangular bounding box of the label as an occupied area in the occupancy bitmap (\autoref{fig:bitmask}).
After placing the label, it continues with the next data point.
The algorithm to add labels in the three chart types mentioned only differs in
(1) the graphical marks to be avoided by labels
and (2) the candidate positions for labels.

In scatter and connected scatter plots, candidate positions of a label are around the graphical point mark it represents.
For a scatter plot, the marks to be avoided by labels include the point marks that represent data points in the plot.
For a connected scatter plot, the marks to be avoided by labels include the point marks that represent data points in the plot and the line marks that connect the points (\autoref{fig:connected_scatter}).

In line charts, each line is composed of a series of data points and a line that connects them all as a single path.
Line charts are similar to connected scatter plots, but often one label represents a whole line instead of a data point.
Therefore, the labeling algorithm places one label per line.
Candidate positions of a label for a line are around each data point that represents the line.
The marks to be avoided by labels include the line marks.

A cartographic map (\autoref{fig:eval_airport_vis}) contains geographical locations that need to be annotated by labels and geographic properties that need to be avoided by the labels.
Candidate positions of a label are (similar to scatterplots) around the locations to annotate.
The marks to be avoided by labels include the point marks that represent locations and marks that represent geographical features (\eg country outlines and paths).
