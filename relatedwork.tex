\section{Related Work}

Prior work on automatic label placement has investigated different aspects of labeling including the optimization goal of the labeling algorithm, the method to detect overlapping marks, label positioning, priority of each label, and orientation of each label.

Existing approaches for placing labels often either prioritize visual quality %, the number of labels placed, 
or runtime performance.
Several projects aimed to improve visual quality of certain chart types by defining and optimizing certain quality metrics~\cite{ladislav:layout-aware, cmolik:ghost,timo:agent,kouril:level,wu:zone}.
However, these approaches are not generalizable as these quality metrics are typically specific 
to the chart types. 
As the number of labels placed is important for giving more information to readers,
the number of labels placed is often used as a proxy for visual quality. 
Some has applied several techniques such as simulated annealing \cite{zoraster:annealing} and
0-1 integer programming \cite{zoraster:int-program} to increase the number of labels placed.
However, these approaches are slow as they iteratively adjust label layouts for better ones.
To achieve interactive runtime performance, prior works use a greedy approach \cite{luboschik:particle, mote:informed-greedy}.
These algorithms can place 10,000 labels within the order of milliseconds.
Therefore, they are suitable for visualizing large data sets or interactive charts.

In general, a greedy label placement algorithm has two inputs:
(1) a set of data points $D$ to label,
(2) a set of existing marks $M$ that labels need to avoid.
From these inputs, it takes the following steps to determine label placements:
\begin{enumerate}
  \vspace{-5pt}
  \item Include all the marks $M$ in a data structure $O$ that stores occupancy information.
  \vspace{-5pt}
  \item For each data point in $D$:
  \vspace{-5pt}
  \begin{enumerate}
    \item Determines a list of candidate positions $P$ nearby its corresponding marks, ordered by their preferences.
    \vspace{-2.5pt}
    \item Find the most preferable position $p \in P$ that does not overlap with any mark as recorded in $O$.
    \vspace{-2.5pt}
    \item If a non-overlapping position $p$ exists, place the label at the position $p$ and update $O$ to include the label placed.
  \end{enumerate}
\end{enumerate}

To determine candidate label positions for a mark, labeling algorithms often use 8-position model~\cite{imhof1975positioning}, generating candidate positions based on the four corners (e.g., top-left) and sides (top, bottom, left, and right) of the mark's axis-aligned bounding rectangle.
Hirsch~\cite{hirsch1982algorithm} extends this discrete positioning approach as a more generalized "slider model".
This paper applies the standard 8-position model to generate candidate positions for different chart types 
and focus on accelerating overlap detection.

Since detecting overlapping marks is the bottleneck for label placement algorithms, prior work has investigated
data structures to speed up overlap detection.
The \emph{trellis strategy} by Mote \ea \cite{mote:informed-greedy} subdivides a chart into a two dimensional grid.
To check if a label can be placed at a position, it checks the positions of other data points and their labels in neighboring grid boxes.

To generalize trellis strategy for arbitrary marks, Luboschik \ea presents Particle-Based Labeling \cite{luboschik:particle}, 
which represents a mark as a set of \emph{virtual particles} that are sampled to cover the areas occupied by the mark.
It then applies the trellis strategy to check for overlaps between the virtual particles instead of the actual marks.
To sample particles from a mark, they propose two approaches. 
First, image-based sampling rasterizes all the marks in $M$ onto an image and then samples particles 
from occupied pixels. 
Alternatively, the vector-based approach samples points to represent the contours of vector graphics of marks.

Particle-Based Labeling works for any kind of marks, but it is more efficient for detecting overlaps between labels and small marks. %  such as overlaps between points in a scatter plot or between labels. 
For large filled marks (such as an area in area chart), Particle-Based Labeling can be inefficient 
because it needs to represent a filled mark with many particles densely placed inside the mark's occupied area.
Thus, checking whether the position of a label is occupied by any mark in a particular grid box is expensive.
This paper presents a Bitmap-Based algorithm, which improves upon Particle-Based Labeling and can efficiently detect overlaps in charts with large filled graphical marks.
