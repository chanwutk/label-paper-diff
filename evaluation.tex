\section{Evaluation}

To evaluate our labeling algorithm using \emph{occupancy bitmap}, we compare our algorithm to the Particle-Based Labeling algorithm from Luboschik~\ea \cite{luboschik:particle}, the current state-of-the-art fast labeling algorithm.
The criteria that we use to compare are (1) runtime and (2) visual quality of the label placements.
We use the number of labels placed as an indication for visual quality because more labels give more information to readers.
To facilitate the comparison, we implemented both algorithms as transforms in Vega~\cite{satyanarayan:vega}.

The task that we use to benchmark our algorithm is to label a map that shows airports in the US
and travel routes between the Seattle-Tacoma airport (Sea-Tac) and other airports\footnote{This map is originally from the Vega-Lite example gallery at \url{https://vega.github.io/vega-lite/examples/geo_rule.html}.}, as shown in \autoref{fig:eval_airport_vis}.
The dataset contains 3320 airports and 56 routes from Sea-Tac.
In the chart, each black dot represents an airport with a route to Sea-Tac.
A black line between the airport and Sea-Tac depicts the corresponding route.
Red texts each in a red box are the labels representing names of airports that have a direct route to Sea-Tac.
Meanwhile, a gray dot represents an airport without a direct route to Sea-Tac.
The chart also outlines states in the US in light gray.
In this benchmark task, we run the algorithms to place labels (shown in teal) for airports without a direct route to Sea-Tac.
Each airport contains eight candidate label positions (2 horizontal, 2 vertical, and 4 diagonal) around the airport location.
The lines, points, red labels, and outline paths are positioned before running the algorithm.
They act as obstacles for the teal labels to avoid.

To account for higher resolution displays, we test the algorithm with chart widths from $1000$ up to $8000$.
The benchmark uses the image-based approach to project the graphical marks to be avoided by labels.
We choose to use the image-based over the vector-based approach for two reasons.
First, the image projection is easy to implement with existing image libraries.
The vector-based approach needs to be re-implemented for each mark type.
Second, the image-based approach is parameter-free.
In contrast, the vector based approach requires adjusting the sampling rate of particles to balance the projection's accuracy and performance when detecting overlaps.

From Luboschik \ea \cite{luboschik:particle}, we notice that there are two issues when projecting graphical marks.
First, a particle that represents an occupied pixel is placed at the center of the pixel.
This placement of particles allows labels to up to half-way overlap with a pixel that is supposed to be occupied (\autoref{fig:eval_airport_vis}).
Second, the algorithm projects every occupied pixel into a particle, which is unnecessarily too many.
The amount of particles used affects the runtime of the algorithm as overlap detection needs to compare a position to more particles.

We addressed these two issues in a version of Particle-Based Labeling, which we refer to as improved Particle-Based Labeling.
We addressed the first issue, a correctness issue, by placing particles at the four corners of an occupied pixel.
Since a label is a zero-degree-angled rectangle, it cannot overlap with an occupied pixel without overlapping with a particle at one of its corners first.
The second is a runtime issue that we addressed by omitting some particles that are too close to an existing particle.
Projection in our improved algorithm has two parts.
First, it projects all particles along the outlines of graphical marks.
Second, it projects particles inside graphical marks every other $H_{min}$ pixels vertically and $W_{min}$ horizontally pixels.
$H_{min}$ is the height of the label with the shortest height.
$W_{min}$ is the width of the label with the shortest width.
This optimization retains the algorithm's correctness, while greatly reducing the number of particles placed.

\subsection{Performance}

For each parameter (labeling algorithm and chart's width), we run the task 20 times and take their median value.
The result in \autoref{fig:eval_airport_performance} shows that
the improved Particle-Based Labeling algorithm is faster than the original one as the chart size increases.
Our bitmap-based algorithm performs significantly better than both the original and the improved Particle-Based Labeling algorithm.
Our algorithm takes at least 22\% less time to run, and the improvement tends to be higher as the chart size increases.

\subsection{Visual Quality}

The original Particle-Based Labeling algorithm places significantly more labels than our improved version.
This means that its projection process of marks to be avoided by labels largely contributes to the increase of placed labels.
Therefore, these excess labels overlap with one of the graphical marks in the chart as shown in \autoref{fig:eval_airport_vis} (D).
The improved version can catch these overlaps because its particles placed at each pixel's corners cover the whole pixel.
The original version cannot catch these overlaps because its placement of a particle at each pixel's center does not cover the whole pixel.

Comparing our algorithm to the improved Particle-Based Labeling algorithm, our algorithm placed 0.8\% fewer labels in the chart with a width of 8000, and 3.2\% fewer labels at a width of 1000.
Our algorithm is able to place a similar number of labels as the Particle-Based Labeling algorithm if we only count labels that do not overlap with any marks.
