\section{Evaluation}

To evaluate our labeling algorithm using \emph{occupancy bitmap}, we compare it to Particle-Based Labeling \cite{luboschik:particle}, the current state-of-the-art fast labeling algorithm.
To perform this comparison, we implemented both algorithms as transforms in Vega~\cite{satyanarayan:vega}
and measure runtime and number of labels placed for each condition.

Our benchmark example is a map that shows airports in the US
and travel routes between the Seattle-Tacoma airport (Sea-Tac) and other airports\footnote{This map is originally from the Vega-Lite example gallery at \url{https://vega.github.io/vega-lite/examples/geo_rule.html}.}, as shown in \autoref{fig:eval_airport_vis}.
The dataset contains 3320 airports and 56 routes from Sea-Tac.
In the chart, each black dot represents an airport with a route to Sea-Tac.
A black line between the airport and Sea-Tac depicts the corresponding route.
Red texts each in a red box are the labels representing names of airports that have a direct route to Sea-Tac.
Meanwhile, a gray dot represents an airport without a direct route to Sea-Tac.
The chart also outlines US states in light gray.
In this benchmark, we run the algorithms to place labels (shown in teal) for airports without a direct route to Sea-Tac.
Each airport contains eight candidate label positions (2 horizontal, 2 vertical, and 4 diagonal) around the airport location.
The lines, points, red labels, and outline paths are placed before running the algorithm,
acting as obstacles for the teal labels to avoid.
To account for higher resolution displays, we test the algorithm with chart widths ranging from $1000$ pixels up to $8000$ pixels, with a fixed aspect ratio of 5:8.

For the baseline condition, we use the Particle-Based Labeling\cite{luboschik:particle} with image-based sampling instead of vector-based sampling for two reasons.
First, image-based sampling is a more practical approach to adopt in visualization tools because 
every standard graphic library can rasterize any mark types.
Meanwhile, vector-based sampling requires separate implementations for different mark types.
Second, the image-based approach is parameter-free.
In contrast, the vector-based approach requires adjusting the sampling rate of particles to balance the fidelity against runtime performance.

We also notice that the mark rasterization process in Particle-Based Labeling has two issues.
First, a particle that represents an occupied pixel is placed at the center of the pixel.
This placement of particles may allow a label to slightly overlap with other marks by a half pixel, as shown in \autoref{fig:eval_airport_vis}D.
Second, the algorithm rasterizes every occupied pixel into a particle, which is unnecessarily too many.
The number of particles used affects the runtime of the algorithm as overlap detection needs to compare a position to more particles.

We addressed these two issues in a version of Particle-Based Labeling, which we refer to as Improved Particle-Based Labeling.
We addressed the first issue, a correctness issue, by placing particles at the four corners of an occupied pixel.
Since a label's bounding box is an axis-aligned rectangle, it cannot overlap with an occupied pixel without overlapping with a particle at one of its corners first.
We then address the runtime issue by omitting particles that are too close to others and thus are redundant.
To do so, our improved algorithm rasterize a mark in two phases. 
First, it rasterizes all particles along the outlines of the mark.
Second, it rasterizes particles inside the marks for every other $H_{min}$ pixels vertically and $W_{min}$ pixels horizontally,
where $H_{min}$ is the height of the label with the shortest height and $W_{min}$ is the width of the label with the shortest width.
This optimization retains the algorithm's correctness, while greatly reducing the number of particles placed.

\subsection{Performance}

For each experimental condition (labeling algorithm and chart width), we run the task 20 times and 
calculate the median runtime and number of labels placed.
\autoref{fig:eval_airport_performance} shows that
the improved Particle-Based Labeling algorithm is faster than the original one as the chart size increases.
Our Bitmap-Based algorithm performs significantly better than both the original and Improved Particle-Based Labeling algorithms,
taking at least 22\% less time to run across the chart sizes. 
The improvement also generally increases as the chart size increases.

\subsection{Number of Labels Placed}

As we discussed earlier, the original Particle-Based Labeling may allow a label
to overlap with a mark by a half pixel, thus it places significantly more labels than 
Improved Particle-Based Labeling and Bitmap-Based Labeling.

To avoid the effect of this correctness issue, we focus on the comparison of Bitmap-Based Labeling 
with Improved Particle-Based Labeling. 
Bitmap-Based Labeling placed 0.8\% fewer labels for charts with 8000 pixel width 
and 3.2\% fewer labels for charts with 1000 pixel width.
Thus, we can conclude that Bitmap-Based Labeling can place a similar number of labels as Particle-Based Labeling if we only count labels that do not overlap with any marks.
