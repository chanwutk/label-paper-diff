\firstsection{Introduction}
\maketitle

Text labels are important for annotating charts with details of specific data points.
To be legible, labels should not overlap with other graphical marks in the chart.
Since manual label placement can be tedious, prior work proposed automatic label placement algorithms (\eg \cite{luboschik:particle,mote:informed-greedy,wu:zone,zoraster:int-program,zoraster:annealing}).
As the placement of each label can be arbitrary and depend on the placement of other labels in the chart,
perfectly maximizing the number of placements is an NP-hard problem with respect to the number of labels to be placed.
In practice, label placement algorithms need to strike a balance between achieving better performance
(especially for interactive applications) and maximizing the number of labels placed.

To achieve interactive performance, many label placement algorithms (\eg \cite{luboschik:particle,mote:informed-greedy}) use a greedy approach, instead of examining \emph{all} combinations of label placements.
To place each label, these algorithms first determine a list of preferred positions.
They then place each label at a preferred position that is unoccupied.
If all possible placements lead to overlaps, they omit the particular label.
This greedy approach greatly reduces the search space to be linear with respect to the number of labels.
However, detecting overlapping elements remains the bottleneck.
A naïve overlap detection by comparing each position of a label with all placed labels yields an $O(n^2)$ runtime in a chart with $n$ labels~\cite{emden:prism}, which can be problematic for charts with many marks.

Particle-Based Labeling~\cite{luboschik:particle}, the state-of-the-art fast labeling algorithm,
accelerates overlap detection by simulating shapes as particles (collections of points) and
comparing each label position only to particles in its neighborhood.
This approach works well for charts that contain small shapes like scatter plots.
However, for larger shapes, the algorithm needs to sample many points to simulate the shape's form,
significantly increasing required computations for overlap detection.
As the number of points to check with depends on the number of marks and their sizes, the required computation increases significantly for plots with many large marks.

In this paper, we aim to improve the performance of label placement algorithms
with a more efficient way to detect overlapping elements.
In addition, we aim to generalize the overlap detection technique so that it can be used with different types of charts.
To achieve these goals, we make three contributions.

First, we present \emph{occupancy bitmaps}, which record if pixels on a particular chart are occupied, as a new data structure for fast label overlap detection.
All graphical marks are rasterized to a bitmap to record the pixels that they occupy.
This bitmap structure can leverage bitwise operators to quickly detect if a new label overlaps with any existing elements in the chart and update occupancy information after a new label is placed on the chart.
With this approach, the cost to detect overlaps for a new label is fixed based on the chart size and size of the label, regardless of the number and the size of other graphical marks in the chart.

Second, we apply occupancy bitmaps to label various charts with different placement strategies including scatter plots, connected scatter plots, line charts, and cartographic maps.

Third, to evaluate our approach, we compare it to Particle-Based Labeling~\cite{luboschik:particle}.
Our approach requires over \emph{22\% less} time to label a map of 3320 airports in the US and reachable airports from SEA-TAC airport,
while placing a comparable number of labels.
To facilitate this evaluation and the adoption of our method, we implement it as an extension to the Vega visualization tool~\cite{satyanarayan:vega}.
