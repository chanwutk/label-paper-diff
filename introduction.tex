\firstsection{Introduction}
\maketitle

Text labels are important for annotating charts with details of specific data points.
To be legible, labels should not overlap with other graphical marks in the chart.
Since manual label placement can be tedious, prior work proposed automatic label placement algorithms (\eg \cite{luboschik:particle,mote:informed-greedy,zoraster:int-program,zoraster:annealing}).
As the placement of each label can be arbitrary and depend on the placement of other labels in the chart,
perfectly maximizing the number of placements is an NP-hard problem with respect to the number of labels to be placed.
In practice, label placement algorithms need to strike a balance between achieving better performance
(especially for interactive applications) and maximizing the number of labels placed.

To achieve interactive performance, many label placement algorithms (\eg \cite{luboschik:particle,mote:informed-greedy}) use a greedy approach, instead of examining \emph{all} combinations of label placements.
To place each label, these algorithms first determine a list of candidate placements.
They then place each label at one of the positions without overlapping with any labels placed earlier.
If all possible placements lead to overlap, they omit the particular label.
This greedy approach greatly reduces the search space to be linear with respect to the number of labels.
Now, detecting overlapping elements remains the bottleneck.
A naïve overlap detection by comparing each position of a label with all placed labels yields an $O(n^2)$ runtime in a chart with $n$ labels.
Its runtime still does not scale well for charts with many marks.

Particle-Based Labeling~\cite{luboschik:particle}, the state-of-the-art fast labeling algorithm,
accelerates overlap detection by simulating shapes as particles (collections of points) and
comparing each label position only to particles in its neighborhood.
This approach works well for charts that contain small shapes like scatter plots.
However, for larger shapes, the algorithm needs to sample many points to simulate the shape's form,
significantly increasing required computations for overlap detection.
As the number of points to check with depends on the number of marks and their sizes, the required computation increases significantly for plot with many large marks.

In this paper, we aim to improve the performance of label placement algorithms
with a more efficient way to detect overlapping elements.
In addition, we also aim to generalize the overlap detection technique so that it can be used with different types of charts.
To achieve these goals, we make three contributions.

First, we present \emph{occupancy bitmaps}, which record for each pixel on a particular chart whether it is occupied, as a new type of data structure for fast label overlap detection.
All graphical marks are projected to a bitmap to record the area of pixels that they occupy.
This bitmap structure leverages bitwise operators to quickly detect if a new label overlaps with any existing elements in the chart and update occupancy information after a new label is placed on the chart.
With this approach, the cost to detect overlaps for a new label is fixed based on the chart size and size of the label, regardless of the number and the size of other graphical marks in the chart.

Second, we apply occupancy bitmaps to label various charts with different mark types and placement strategies.
We apply them to place labels (\ie position them around data points) in scatter plots, line charts, cartographic maps, and their combinations.

Third, to evaluate our approach, we compare it to Particle-Based Labeling.
Our approach requires over 22\% \emph{less} time to label a map of 3320 airports in the US and reachable airports from SEA-TAC airport,
while producing labels with comparable visual quality measured as the number of labels placed.
To facilitate this evaluation and the adoption of our method, we implement it as an extension to the Vega visualization tool~\cite{satyanarayan:vega}.
