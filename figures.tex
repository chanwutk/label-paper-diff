\newcommand{\figurePositions}{
  \begin{figure}[tb]
    \centering
      \includegraphics[width=\columnwidth]{figures/positions.png}
    \caption{
      4-position model is represented by (A). 8-position model is represented by (A) and (B). 4-slider model is represented by (C).
    }
    \vspace{-10pt}
    \label{fig:positions}
  \end{figure}
}

\newcommand{\figurePlaceLabel}{
  \begin{figure}[tb]
    \centering
    \includegraphics[width=.28\columnwidth]{figures/prefill_bitmap_2}
    \includegraphics[width=.01\columnwidth]{figures/spacer}
    \includegraphics[width=.33\columnwidth]{figures/place_label_2.png}
    \includegraphics[width=.01\columnwidth]{figures/spacer}
    \includegraphics[width=.33\columnwidth]{figures/place_label_4}
    \caption{
      (Left) the orange pixels are the projection of the connected scatter plot into the bitmap to mark those pixels as occupied.
      (Middle) When placing a label for a data point, the eight rectangles are the eight candidate positions for the label to be placed.
      The green positions are available, while the red ones are not.
      (Right) After placing the label ``1975'', the pixels under the label need to be marked as occupied.
    }
    \vspace{-10pt}
  \label{fig:place_label}
  \end{figure}
}

\newcommand{\figureBitmask}{
  \begin{figure}[tb]
    \centering
    \includegraphics[width=\columnwidth]{figures/bitmask-short.png}
    \caption{
      The black indices indicate the x/y coordinate of pixels in the chart.
      The red indices indicate the indices of the underlying array of the bitmap.
      For the purpose of demonstration, the bitmap is implemented on an array of 4-bit integers each representing a bit-string of length 4.
      The blue circles are marking occupied pixels.
      The yellow box is the area to lookup or update.
    }
  \label{fig:bitmask}
  \end{figure}
}

\newcommand{\figureConnectedScatter}{
  \begin{figure}[tb]
    \centering
      \includegraphics[width=.4\columnwidth]{figures/connectec-scatter.png}
      \includegraphics[width=.05\columnwidth]{figures/spacer.png}
      \frame{\includegraphics[width=.4\columnwidth]{figures/connected-scatter-bitmap.png}}
    \caption{
      (Left) Labeled connected scatter plot.
      (Right) A snapshot of the bitmap when labeling the connected scatter plot. In this figure, a greedy labeling algorithm already placed labels in the left half of the chart.
    }
    \vspace{-10pt}
    \label{fig:connected_scatter}
  \end{figure}
}

\newcommand{\figureLineChart}{
  \begin{figure}[tb]
    \centering
      \includegraphics[width=.45\columnwidth]{figures/line_chart.png}
      \includegraphics[width=.05\columnwidth]{figures/spacer.png}
      \frame{\includegraphics[width=.45\columnwidth]{figures/line_chart_bitmap.png}}
    \caption{
      (Left) Each line in the chart represents $CO_2$ level in each decade.
      The decade number is shown as a label to the right end of each line.
      In this chart we choose the only anchor point of each line to be the right most point of the line, and anchor points to be to the right, top-right, bottom-right of the anchor point.
      For the label "196x" and "197x", both are labeled at the  bottom-right position.
      The right positions make them overlap with the label "195x" and "196x" respectively.
      (Right) The bitmap result after all the labels are placed.
    }
    \vspace{-10pt}
    \label{fig:line_chart}
  \end{figure}
}

\newcommand{\figureStackedBarChart}{
  \begin{figure}[tb]
    \centering
      \includegraphics[width=.45\columnwidth]{figures/stacked_bar.png}

      \frame{\includegraphics[width=.45\columnwidth]{figures/stacked_bar_bitmap.png}}
      \includegraphics[width=.05\columnwidth]{figures/spacer.png}
      \frame{\includegraphics[width=.45\columnwidth]{figures/stacked_bar_bitmap_border.png}}
    \caption{
      (Top) A stacked bar chart where each bar is labeled with its height.
      The labels are placed by our algorithm where the only position option for label is at the top inside of the bar it represents.
      (Bottom-Left) Bitmap of the fills of bars and labels, used in the algorithm.
      Labels are placed inside each bar, so they are not presented in the unoccupied space.
      (Bottom-Right) Bitmap of the strokes of bars and labels, used in the algorithm.
      Four of the labels are hidden (not placed) as each of them does not fit in the bar it represents.
    }
    \vspace{-10pt}
    \label{fig:stacked_bar_chart}
  \end{figure}
}

\newcommand{\figureLabelInside}{
  \begin{figure}[tb]
    \centering
    \includegraphics[width=\columnwidth]{label_inside_2}
    \caption{
      (Top-Left) overlapping objects that this algorithm is labeling.
      (Top-Right) the projection of the objects to the bitmaps; yellow bits are the projection of the actual fill of the objects to the first bitmap; red bits are the the projection of the borders of the objects to the second bitmap.
      (Bottom-Left) The green label can be placed in the position.
      In this case, the algorithm only looks at the overlapping with borders.
      Since this label does not overlap with any border, it can be placed at the position.
      (Bottom-Right) The red label overlaps with borders; therefore, it cannot be placed in the position.
    }
    \vspace{-10pt}
    \label{fig:label_inside_2}
  \end{figure}
}

\newcommand{\figureEvalAirportVis}{
  \begin{figure}[tb]
    \centering
      \includegraphics[width=\columnwidth]{figures/eval_reachable_airports_vis.png}
    \caption{
        The visualizations from an evaluation to compare (A) our algorithm to (B) the Particle-Based Labeling algorithm from Luboschik \ea \cite{luboschik:particle}.
        (C) is the visual difference of (A) and (B).
        (D) One of the labels placed by the Particle-Based Labeling algorithm overlaps with the line that the label is supposed to avoid as indicated with the red cross.
        Our proposed algorithms address these issues.
    }
    \vspace{-10pt}
    \label{fig:eval_airport_vis}
  \end{figure}
}

\newcommand{\figureEvalAirportPerformance}{
  \begin{figure}[tb]
    \centering
      \includegraphics[width=\columnwidth]{figures/eval_reachable_airports_results.png}
    \caption{
      Results of evaluations comparing our algorithm, the original Particle-Based Labeling algorithm, and our improved Particle-Based Labeling algorithm.
      The evaluations compare runtime and number of label placed.
      Gray bands show the difference for each comparison.
    }
    \vspace{-10pt}
    \label{fig:eval_airport_performance}
  \end{figure}
}
