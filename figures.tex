\newcommand{\figurePlaceLabel}{
  \begin{figure}[tb]
    \centering
    \includegraphics[width=.28\columnwidth]{figures/prefill_bitmap_2}
    \includegraphics[width=.01\columnwidth]{figures/spacer}
    \includegraphics[width=.33\columnwidth]{figures/place_label_2.png}
    \includegraphics[width=.01\columnwidth]{figures/spacer}
    \includegraphics[width=.33\columnwidth]{figures/place_label_4}
    \caption{
      (Left) the orange pixels are the projection of the connected scatter plot into the bitmap to mark those pixels as occupied.
      (Middle) When placing a label for a data point, the eight rectangles are the eight candidate positions for the label to be placed.
      The green positions are available, while the red ones are not.
      (Right) After placing the label ``1975'', the pixels under the label need to be marked as occupied.
    }
    \vspace{-10pt}
  \label{fig:place_label}
  \end{figure}
}

\newcommand{\figureBitmask}{
  \begin{figure}[tb]
    \centering
    \includegraphics[width=\columnwidth]{figures/bitmask-short.png}
    \caption{
      The black indices indicate the x/y coordinate of pixels in the chart.
      The red indices indicate the indices of the underlying array of the bitmap.
      For the purpose of demonstration, the bitmap is implemented on an array of 4-bit integers each representing a bit-string of length 4.
      The blue circles are marking occupied pixels.
      The yellow box is the area to lookup or update.
    }
  \label{fig:bitmask}
  \end{figure}
}

\newcommand{\figureConnectedScatter}{
  \begin{figure}[tb]
    \centering
      \includegraphics[width=.4\columnwidth]{figures/connectec-scatter.png}
      \includegraphics[width=.05\columnwidth]{figures/spacer.png}
      \frame{\includegraphics[width=.4\columnwidth]{figures/connected-scatter-bitmap.png}}
    \caption{
      (Left) Labeled connected scatter plot.
      (Right) A snapshot of the bitmap when labeling the connected scatter plot. In this figure, a greedy labeling algorithm already placed labels in the left half of the chart.
    }
    \vspace{-10pt}
    \label{fig:connected_scatter}
  \end{figure}
}

\newcommand{\figureEvalAirportVis}{
  \begin{figure}[tb]
    \centering
      \includegraphics[width=\columnwidth]{figures/eval_reachable_airports_vis.png}
    \caption{
        The visualizations from an evaluation to compare (A) our algorithm to (B) the Particle-Based Labeling algorithm from Luboschik \ea \cite{luboschik:particle}.
        (C) is the visual difference of (A) and (B).
        (D) One of the labels placed by the Particle-Based Labeling algorithm overlaps with the line that the label is supposed to avoid as indicated with the red cross.
        Our proposed algorithms address these issues.
    }
    \vspace{-10pt}
    \label{fig:eval_airport_vis}
  \end{figure}
}

\newcommand{\figureEvalAirportPerformance}{
  \begin{figure}[tb]
    \centering
      \includegraphics[width=\columnwidth]{figures/eval_reachable_airports_results.png}
    \caption{
      Results of evaluations comparing our algorithm, the original Particle-Based Labeling algorithm, and our improved Particle-Based Labeling algorithm.
      The evaluations compare runtime and number of label placed.
      Gray bands show the difference for each comparison.
    }
    \vspace{-10pt}
    \label{fig:eval_airport_performance}
  \end{figure}
}
