Fig 1:
(Left) We rasterize connected scatter plot onto the bitmap to mark occupied pixels, shown in orange.
(Middle) We use the 8-position model~\cite{imhof1975positioning} to generate candidate positions for label placements.
The cyan positions are available, while the red ones are not.
(Right) After placing the label ``1975'', the pixels under the label need to be marked as occupied.

Fig 2:
The black indices indicate the x/y coordinate of pixels in the chart.
The red indices indicate the indices of the underlying array of the bitmap.
For the purpose of demonstration, the bitmap is implemented on an array of 4-bit integers each representing a bit-string of length 4.
The blue circles are marking occupied pixels.
The yellow box is the area to lookup or update.

Fig 3:
(Left) Labeled connected scatter plot.
(Right) A snapshot of the bitmap when labeling the connected scatter plot. Here, a greedy labeling algorithm already placed labels in the left half of the chart.

Fig 4:
The labeling results from (A) our Bitmap-Based Labeling and (B) Particle-Based Labeling by Luboschik \ea \cite{luboschik:particle}.
(C) shows the visual difference between (A) and (B).
The original Particle-Based Labeling may place a label that overlaps with existing marks by a half pixel. For example, the bounding box of the text's bounding box, as indicated with the red cross in (D), overlaps with a nearby line. 
Our Improved Particle-Based Labeling algorithm address this issue.

Fig 5:
The runtime and the number of label placed by the Bitmap-Based algorithm, the original Particle-Based Labeling algorithm, and the Improved Particle-Based Labeling algorithm.
The gray bands show the differences between conditions.
